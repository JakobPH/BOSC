%%%%%%%%%%%%%%%%%%%%%%%%%%%%%%%%%%%%%%%%%
% Programming/Coding Assignment
% LaTeX Template
%
% This template has been downloaded from:
% http://www.latextemplates.com
%
% Original author:
% Ted Pavlic (http://www.tedpavlic.com)
%
% Note:
% The \lipsum[#] commands throughout this template generate dummy text
% to fill the template out. These commands should all be removed when 
% writing assignment content.
%
% This template uses a Perl script as an example snippet of code, most other
% languages are also usable. Configure them in the "CODE INCLUSION 
% CONFIGURATION" section.
%
%%%%%%%%%%%%%%%%%%%%%%%%%%%%%%%%%%%%%%%%%

%----------------------------------------------------------------------------------------
%	PACKAGES AND OTHER DOCUMENT CONFIGURATIONS
%----------------------------------------------------------------------------------------

\documentclass{article}

\usepackage{fancyhdr} % Required for custom headers
\usepackage{lastpage} % Required to determine the last page for the footer
\usepackage{extramarks} % Required for headers and footers
\usepackage[usenames,dvipsnames]{color} % Required for custom colors
\usepackage{graphicx} % Required to insert images
\usepackage{listings} % Required for insertion of code
\usepackage{courier} % Required for the courier font
\usepackage{lipsum} % Used for inserting dummy 'Lorem ipsum' text into the template
\usepackage[utf8]{inputenc}

% Margins
\topmargin=-0.45in
\evensidemargin=0in
\oddsidemargin=0in
\textwidth=6.5in
\textheight=9.0in
\headsep=0.25in

\linespread{1.1} % Line spacing

% Set up the header and footer
\pagestyle{fancy}
\lhead{Jens Egholm Pedersen} % Top left header
\chead{BOSC - Opgave 2} % Top center head
\rhead{Torsdag d. 31. oktober, 2013} % Top right header
\lfoot{} % Bottom left footer
\cfoot{} % Bottom center footer
\rfoot{Side\ \thepage\ af\ \protect\pageref{LastPage}} % Bottom right footer
\renewcommand\headrulewidth{0.4pt} % Size of the header rule
\renewcommand\footrulewidth{0.4pt} % Size of the footer rule

\setlength\parindent{0pt} % Removes all indentation from paragraphs

\lstset{
  numbers=left,
  stepnumber=1,    
  firstnumber=1,
  numberfirstline=true
}

\begin{document}

\section*{BOSC - Opgave 2}
Dette er min løsning til opgaverne i den anden obligatoriske opgave. Kodefilerne er vedhæftet som appendiks. 

\subsection*{Opgave 1}
Følgende output viser en af kørslerne af programmet \texttt{sqrt.c}:
\begin{verbatim}
Thread : 1 	 Speedup: 1.000000 	 Efficiency: 1.000000
Threads: 4 	 Speedup: 1.193548 	 Efficiency: 0.298387
Threads: 8 	 Speedup: 1.913793 	 Efficiency: 0.239224
Threads: 16 	 Speedup: 2.361702 	 Efficiency: 0.147606
Threads: 32 	 Speedup: 3.468750 	 Efficiency: 0.108398
Threads: 64 	 Speedup: 5.285714 	 Efficiency: 0.082589
Threads: 128 	 Speedup: 5.842105 	 Efficiency: 0.045641
Threads: 256 	 Speedup: 5.045455 	 Efficiency: 0.019709
Threads: 512 	 Speedup: 3.264706 	 Efficiency: 0.006376
Threads: 1024 	 Speedup: 2.642857 	 Efficiency: 0.002581
Threads: 2048 	 Speedup: 1.913793 	 Efficiency: 0.000934
Threads: 4096 	 Speedup: 1.353659 	 Efficiency: 0.000330
Threads: 8192 	 Speedup: 0.740000 	 Efficiency: 0.000090
Threads: 16384 	 Speedup: 0.417293 	 Efficiency: 0.000025
\end{verbatim}
Det ses at det maksimale speedup findes ved 128 tråde, som er hele 5.8 gange mere effektivt end kun 1 tråd. 

\subsection*{Opgave 2}
Problemet med flere tråde i dette eksempel er særligt det bogen nævner som 'data dependency': deling af ressourcer. Når en tråd eksempelvis beslutter sig for at lægge et nyt element i listen kan den risikere at blive afbrudt midt i kaldet til \texttt{list\_add}, eksempelvis lige efter den har sat \texttt{last->next}. Den anden tråd der afbryder har også tænkt sig at indsætte et element, men når modsat den første tråd at gennemføre. Når den første tråd fortsætter, har den anden tråd overskrevet referencen til \texttt{next} i det sidste element i listen, men det ved den første tråd jo ikke, så den indsætter elementet i halen. Vi har nu en inkonsistent liste, hvis haleelement ikke har en pegepind fra det forrige element i listen.

Den færdige implementation kan findes i bilag \ref{listcode}.

\subsection*{Opgave 3}
Som beskrevet i opgaven har jeg skrevet programmet til at tage fire inputs: antal producers, antal consumers, størrelsen af den delte buffer og antal produkter der skal produceres. Alle producers og consumers kører i deres egen tråd hvor de 'sover' hver gang de producerer eller konsumerer. Pga. semaphorerne er der ingen starvation eller deadlocks, og programmet venter til sidst på alle tråde før det slutter. Programkoden findes i appendiks \ref{prodconscode}.

En kørsel af programmet kan ses herunder:
\begin{verbatim}
% ./prodcons 6 6 10 9 
Initialising 6 producers and 6 consumers with a buffer size of 10
Producer   5 produced  Elm 5. Items in buffer:   1 (out of  10).
Consumer   1 consumed  Elm 5. Items in buffer:   0 (out of  10).
Producer   5 produced  Elm 4. Items in buffer:   1 (out of  10).
Consumer   4 consumed  Elm 4. Items in buffer:   0 (out of  10).
Producer   3 produced  Elm 3. Items in buffer:   1 (out of  10).
Consumer   5 consumed  Elm 3. Items in buffer:   0 (out of  10).
Producer   2 produced  Elm 1. Items in buffer:   1 (out of  10).
Consumer   2 consumed  Elm 1. Items in buffer:   0 (out of  10).
Producer   0 produced  Elm 6. Items in buffer:   1 (out of  10).
Consumer   0 consumed  Elm 6. Items in buffer:   0 (out of  10).
Producer   5 produced  Elm 8. Items in buffer:   1 (out of  10).
Consumer   3 consumed  Elm 8. Items in buffer:   0 (out of  10).
Producer   2 produced  Elm 2. Items in buffer:   1 (out of  10).
Producer   3 produced  Elm 9. Items in buffer:   2 (out of  10).
Producer   5 produced  Elm 7. Items in buffer:   3 (out of  10).
Consumer   1 consumed  Elm 2. Items in buffer:   2 (out of  10).
Consumer   4 consumed  Elm 9. Items in buffer:   1 (out of  10).
Consumer   5 consumed  Elm 7. Items in buffer:   0 (out of  10).
\end{verbatim}

\subsection*{Opgave 4}
Jeg allokerer resourcer fra linje 172 vha. \texttt{malloc} og tjekker derefter i linje 232 (\texttt{issafe()} i linje 136) at vi er i en sikker tilstand. Resourcerne bliver frigivet igen i linje 252.

\texttt{resource\_request} metoden er skrevet ved først at undersøge om vi bliver efterladt i en sikker tilstand efter forespørgslen er udført. Hvis vi gør allokeres resourcerne, hvis ikke returnerer vi 0 for 
\texttt{false}. Før nogle operationer på \texttt{s} bliver udført, låses \texttt{state\_mutex}. Så snart de er færdige låses den op igen.

\texttt{resource\_release} er implementeret ved simpelt nok at fjerne resourcerne igen. Der er også her brugt en mutex til at sikre at vi er alene om vores data i den kritiske zone. Dermed undgår vi også race conditions fordi vi er sikre på at når låsen opnåes, vil den frigives igen efter et stykke tid.

\newpage

\section*{Appendiks}
Jeg har inkluderet \texttt{sqrt.c}, \texttt{list.c}, \texttt{prodcons.c}, \texttt{banker.c} samt \texttt{Makefile} i dette appendiks, som er de eneste filer hvori jeg har tilføjet funktionalitet.

\label{sqrtcode}
\section*{Appendiks A - sqrt.c}
\lstinputlisting[language=C]{sqrt.c}

\newpage
\label{listcode}
\section*{Appendiks B - list.c}
\lstinputlisting[language=C]{list/list.c}

\newpage
\label{prodconscode}
\section*{Appendiks C - prodcons.c}
\lstinputlisting[language=C]{prodcons.c}

\newpage
\label{bankercode}
\section*{Appendiks D - banker.c}
\lstinputlisting[language=C]{banker/banker.c}

\newpage 

\section*{Appendiks E - Makefiles}
\subsection*{Makefile for \texttt{sqrt} and \texttt{prodcons}}
\lstinputlisting[language=make]{Makefile}
\subsection*{Makefile for \texttt{list}}
\lstinputlisting[language=make]{list/Makefile}
\subsection*{Makefile for \texttt{banker}}
\lstinputlisting[language=make]{banker/Makefile}

\label{lastPage}

\end{document}
